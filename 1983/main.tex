\documentclass[12pt,-letter paper]{article}
\usepackage{siunitx}
\usepackage{setspace}
\usepackage{gensymb}
\usepackage{xcolor}
\usepackage{caption}
%\usepackage{subcaption}
\doublespacing
\singlespacing
\usepackage[none]{hyphenat}
\usepackage{amssymb}
\usepackage{relsize}
\usepackage[cmex10]{amsmath}
\usepackage{mathtools}
\usepackage{amsmath}
\usepackage{commath}
\usepackage{amsthm}
\interdisplaylinepenalty=2500
%\savesymbol{iint}
\usepackage{txfonts}
%\restoresymbol{TXF}{iint}
\usepackage{wasysym}
\usepackage{amsthm}
\usepackage{mathrsfs}
\usepackage{txfonts}
\let\vec\mathbf{}
\usepackage{stfloats}
\usepackage{float}
\usepackage{cite}
\usepackage{cases}
\usepackage{subfig}
%\usepackage{xtab}
\usepackage{longtable}
\usepackage{multirow}
%\usepackage{algorithm}
\usepackage{amssymb}
%\usepackage{algpseudocode}
\usepackage{enumitem}
\usepackage{mathtools}
%\usepackage{eenrc}
%\usepackage[framemethod=tikz]{mdframed}
\usepackage{listings}
%\usepackage{listings}
\usepackage[latin1]{inputenc}
%%\usepackage{color}{   
%%\usepackage{lscape}
\usepackage{textcomp}
\usepackage{titling}
\usepackage{hyperref}
%\usepackage{fulbigskip}   
\usepackage{tikz}
\usepackage{graphicx}
\lstset{
  frame=single,
  breaklines=true
}
\let\vec\mathbf{}
\usepackage{enumitem}
\usepackage{graphicx}
\usepackage{siunitx}
\let\vec\mathbf{}
\usepackage{enumitem}
\usepackage{graphicx}
\usepackage{enumitem}
\usepackage{tfrupee}
\usepackage{amsmath}
\usepackage{amssymb}
\usepackage{mwe} % for blindtext and example-image-a in example
\usepackage{wrapfig}
\graphicspath{{figs/}}
\providecommand{\mydet}[1]{\ensuremath{\begin{vmatrix}#1\end{vmatrix}}}
\providecommand{\myvec}[1]{\ensuremath{\begin{bmatrix}#1\end{bmatrix}}}
\providecommand{\cbrak}[1]{\ensuremath{\left\{#1\right\}}}
\providecommand{\brak}[1]{\ensuremath{\left(#1\right)}}
\begin{document}
\begin{enumerate}

 \subsection*{Twenty-fourth International Olympiad,1983}
\item Find all functions $f$ defined on the set of positive real numbers which take positive real values and satisfy the conditions:\\
$\brak{i}$ $f\brak{xf\brak{y}} = yf\brak{x}$ for all positive $x,y$;\\
$\brak{ii}$ $f\brak{x} \rightarrow 0 $ as $ x \rightarrow \infty$.
\item let $A$ be one of the two distinct points of intersection of two unequal coplanar tangents to the circles $C_1$ and $C_2$ with centers $O_1$ and $O_2$, respectively. One of the common tangents to the circles touches $C_1$ at $P_1$ and $C_2$ at $P_2$, while the other touches $C_1$ at $Q_1$ and $C_2$ at $Q_2$.  Let $M_1$ be the midpoint of $P_1Q_1$, $M_2$ be the midpoint of $P_2Q_2$ prove that $\angle O_1AO_2 = \angle M_1AM_2$.
\item Let $a,b$ and $c$ be positive integers, no two of which have a  common divisor grater than $1$. Show that $2abc-ab-bc-ca$ is the largest integer which cannot be expressed in the form $xbc+yca+zab$, where $x,y$ and $z$ are non-negative integers.	
\item Let $ABC$ be an equilateral triangle and $\epsilon$ the set of all points contained in the three segments $AB$, $BC$, and $CA$ (including $A$, $B$, and $C$). Determine whether for every partition of $\epsilon$ into two disjoint subsets, at least one of the two subsets that contains the vertices of a right-angled triangle. Justify your answer.
\item Is it possible to choose $1983$ distinct positive integers, all less than or equal to $10^5$, no three of which are consecutive terms of an arithmetic progression ? justify your answer.	
\item Let $a,b$ and $c$ be the lengths of the sides of a triangle. Prove that.
	\begin{align*} a^2b \brak {a-b}+b^2c\brak{b-c}+c^2a\brak{c-a}\geq 0 \end{align*}
		\\Determine when quality occurs.










\end{enumerate}
\end{document}
