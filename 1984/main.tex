\documentclass[12pt,-letter paper]{article}
\usepackage{siunitx}
\usepackage{setspace}
\usepackage{gensymb}
\usepackage{xcolor}
\usepackage{caption}
%\usepackage{subcaption}
\doublespacing
\singlespacing
\usepackage[none]{hyphenat}
\usepackage{amssymb}
\usepackage{relsize}
\usepackage[cmex10]{amsmath}
\usepackage{mathtools}
\usepackage{amsmath}
\usepackage{commath}
\usepackage{amsthm}
\interdisplaylinepenalty=2500
%\savesymbol{iint}
\usepackage{txfonts}
%\restoresymbol{TXF}{iint}
\usepackage{wasysym}
\usepackage{amsthm}
\usepackage{mathrsfs}
\usepackage{txfonts}
\let\vec\mathbf{}
\usepackage{stfloats}
\usepackage{float}
\usepackage{cite}
\usepackage{cases}
\usepackage{subfig}
%\usepackage{xtab}
\usepackage{longtable}
\usepackage{multirow}
%\usepackage{algorithm}
\usepackage{amssymb}
%\usepackage{algpseudocode}
\usepackage{enumitem}
\usepackage{mathtools}
%\usepackage{eenrc}
%\usepackage[framemethod=tikz]{mdframed}
\usepackage{listings}
%\usepackage{listings}
\usepackage[latin1]{inputenc}
%%\usepackage{color}{   
%%\usepackage{lscape}
\usepackage{textcomp}
\usepackage{titling}
\usepackage{hyperref}
%\usepackage{fulbigskip}   
\usepackage{tikz}
\usepackage{graphicx}
\lstset{
  frame=single,
  breaklines=true
}
\let\vec\mathbf{}
\usepackage{enumitem}
\usepackage{graphicx}
\usepackage{siunitx}
\let\vec\mathbf{}
\usepackage{enumitem}
\usepackage{graphicx}
\usepackage{enumitem}
\usepackage{tfrupee}
\usepackage{amsmath}
\usepackage{amssymb}
\usepackage{mwe} % for blindtext and example-image-a in example
\usepackage{wrapfig}
\graphicspath{{figs/}}
\providecommand{\mydet}[1]{\ensuremath{\begin{vmatrix}#1\end{vmatrix}}}
\providecommand{\myvec}[1]{\ensuremath{\begin{bmatrix}#1\end{bmatrix}}}
\providecommand{\cbrak}[1]{\ensuremath{\left\{#1\right\}}}
\providecommand{\brak}[1]{\ensuremath{\left(#1\right)}}
\begin{document}
\begin{enumerate}


		\subsection*{Twenty-fifth International Olympiad,1984}
\item Prove that $0 \leq yz+zx+xy-2xyz \leq \frac{7}{27}$, $x,y$ and $z$ are non-negaive real numbers for which $x+y+z=1$.
\item Find one pair of positive integers $a$ and $b$ such that :\\
	$\brak{i}$ $ab\brak{a+b}$ is not divisible by $7$;\\
		$\brak{ii}$$\brak{a+b}^7-a^7-b^7$ is divisible by $7^7$\\
		Justify your answer.
	\item In the plane two different points $O$ and $A$ are given. For each point $X$ of the plane, other than $O$, denote by $a\brak{X}$ the measure of the angle between $OA$ and $OX$ in radians countrclockwise from $OA\brak {O\leq a\brak{X}<2\pi}$. Let $C\brak{X}$ be the circle with center $O$ and radius of length $\frac {OX+a\brak{X}}{OX}$. each point  of the plane is colored by one of a finite number of colors. Prove that there exists a point $Y$ for which $a\brak{y}>0$ such that color appears on  the circumference of the circle $C\brak{Y}$.
\item Let $ABCD$ be a convex quadrilateral such tha the line $CD$ is a tangent to the circle on $AB$ as diameter. Prove that the line $AB$ is a tangent to the  circle on $CD$ as diameter if and only if the lines $BC$ and $AD$ are parallel.
\item Let $d$ be the sum of the lengths of all the diagonals of a plane convex polygon with $n$ vertices $\brak{n>3}$,and let $p$ be its perimeter.Prove that.\begin{align*}
n-3<\frac{2d}{p}<\myvec{\frac{n}{2}}\myvec{\frac{n+1}{2}}-2,\end{align*} 
		Where $\myvec{x}$ denotes the gratest integer not exceeding $x$
\item Let $a,b,c$ and $d$ be odd integers such that  $0<a<b<c<d$ and $ad=bc$. Prove that if $a+d=2^k$ and $b+c=2^m$ for some integers $k$ and $m$, then $a=1$ 	


\end{enumerate}
\end{document}

