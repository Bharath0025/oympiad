\documentclass[12pt,-letter paper]{article}
\usepackage{siunitx}
\usepackage{setspace}
\usepackage{gensymb}
\usepackage{xcolor}
\usepackage{caption}
%\usepackage{subcaption}
\doublespacing
\singlespacing
\usepackage[none]{hyphenat}
\usepackage{amssymb}
\usepackage{relsize}
\usepackage[cmex10]{amsmath}
\usepackage{mathtools}
\usepackage{amsmath}
\usepackage{commath}
\usepackage{amsthm}
\interdisplaylinepenalty=2500
%\savesymbol{iint}
\usepackage{txfonts}
%\restoresymbol{TXF}{iint}
\usepackage{wasysym}
\usepackage{amsthm}
\usepackage{mathrsfs}
\usepackage{txfonts}
\let\vec\mathbf{}
\usepackage{stfloats}
\usepackage{float}
\usepackage{cite}
\usepackage{cases}
\usepackage{subfig}
%\usepackage{xtab}
\usepackage{longtable}
\usepackage{multirow}
%\usepackage{algorithm}
\usepackage{amssymb}
%\usepackage{algpseudocode}
\usepackage{enumitem}
\usepackage{mathtools}
%\usepackage{eenrc}
%\usepackage[framemethod=tikz]{mdframed}
\usepackage{listings}
%\usepackage{listings}
\usepackage[latin1]{inputenc}
%%\usepackage{color}{   
%%\usepackage{lscape}
\usepackage{textcomp}
\usepackage{titling}
\usepackage{hyperref}
%\usepackage{fulbigskip}   
\usepackage{tikz}
\usepackage{graphicx}
\lstset{
  frame=single,
  breaklines=true
}
\let\vec\mathbf{}
\usepackage{enumitem}
\usepackage{graphicx}
\usepackage{siunitx}
\let\vec\mathbf{}
\usepackage{enumitem}
\usepackage{graphicx}
\usepackage{enumitem}
\usepackage{tfrupee}
\usepackage{amsmath}
\usepackage{amssymb}
\usepackage{mwe} % for blindtext and example-image-a in example
\usepackage{wrapfig}
\graphicspath{{figs/}}
\providecommand{\mydet}[1]{\ensuremath{\begin{vmatrix}#1\end{vmatrix}}}
\providecommand{\myvec}[1]{\ensuremath{\begin{bmatrix}#1\end{bmatrix}}}
\providecommand{\cbrak}[1]{\ensuremath{\left\{#1\right\}}}
\providecommand{\brak}[1]{\ensuremath{\left(#1\right)}}
\begin{document}
\begin{enumerate}
		\subsection*{Twenty-sixth International Olympiad,1985}
\item $A$ circle has center on the side $AB$ of the cyclic quadrilateral $ABCD$. The other three sides are tangent to the circle. Prove that $AD+BC = AB$.
\item Let  $n$ and $k$ be given relatively prime natural numbers $k<n$. Each number in the set $M={1,2,...n-1}$ is colored either blue or white. It is given that\\
	$\brak{i}$ for each $i  \epsilon   M$, both $i$ and $n-i$ have the same color;\\
		$\brak{ii}$ for each $i  \epsilon  M$, $i\neq k$, both $i$ and $\mydet {i-k}$ have the same color. Prove that all numbers in $M$ must have the same color.
	\item For any polynomial $P\brak{x} = a_0 + a_1x + ..... + a_kx^k$ with integer coefficients,  the number of coefficients which are odd is denoted by $w\brak{P}$.  For $i = 0, 1, ..., $let $Q_i\brak{x} = \brak{1+x}^i$. Prove that if $i_1i_2, ..., i_n$ are integers such
		that $0\leq i_1<i_2<......<i_n$, then \begin{align*}  w(Q_{i1}+Q_{i2},+....+Q_{in})\geq w (Q_{i1}) \end{align*}

\item Given a set $M$ of $1985$ distinct positive integers, none of which has a prime divisor grater than $26$. Prove that $M$ contains at least one subset of four distinct elements whose product is the fourth power of an integer.
\item A circle with center $O$ passes through the vertices $A$ and $C$ of triangle $ABC$ and intersects the segments $AB$ and $BC$ again at distinct points $K$ and $N$ respectively. The circumscribed circle of the triangle $ABC$ and $EBN$ intersect at exactly two distinct points $B$ and $M$. Prove that angle $OMB$ is a right angle.






\item For every real number $x_1$, construct the sequence $x_1, x_2, ... $by setting
	\begin{align*} x_{n+1}=x_n\brak{x_n+\frac{1}{4}}\end{align*} for each $n \geq 1$
Prove that there exists exactly one value of $x_1$ for which
\begin{align*}
    0 < x_n<x_{n+1}<1
\end{align*}
for every $n$.





\end{enumerate}
\end{document}
